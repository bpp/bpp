
\title{Manual for tip dating in bpp}

\date{\today}
\author{Anna Nagel, Tom\'{a}\v{s} Flouri,  Ziheng Yang, and Bruce Rannala}

\documentclass[11pt]{article}
\usepackage{amsmath}
\usepackage{graphicx}
\usepackage{pdflscape}
\usepackage{array}
\usepackage{caption}
\usepackage[margin=.5in]{geometry}
\usepackage{fancyvrb}
\usepackage{fvextra}
\usepackage{hyperref}

\usepackage{tikz}
\usepackage{fullpage}
% \usepackage{comment}
% \usetikzlibrary{calc,intersections,positioning,decorations.pathreplacing,arrows.meta}
\usepackage{listings} \usepackage{xcolor}

% \usepackage{mathptmx}
\usepackage[normalem]{ulem} \usepackage{amssymb,amsmath}
\usepackage{booktabs} \usepackage{natbib} \usepackage{titlesec}
\usepackage{dirtytalk} \usepackage{hyperref} \usepackage{booktabs}
\usepackage{float} \usepackage{mdframed} \usepackage{changepage}

\begin{document}
\maketitle


\section{New Control File options in bpp}\label{control}
Options in the control file that are either new or modified when using tip dating are described below.
For options that are unchanged, we refer readers to the full bpp manual for a full description of all of the options.
\medskip\newline
\subsection{BPP control file variables} \label{controlfilevariables}
\textbf{{\large 1}} \\
\noindent\rule{\textwidth}{0.8pt} \\
\textbf{{\Large \texttt{datefile = s}}} \vspace{5pt}\\
\textbf{DESCRIPTION} \vspace{5pt}\\
Sets the path/name of the date file to be the string s\vspace{5pt}\\
\textbf{VALUES} \vspace{5pt}\\
s, a string of characters specifying the directory path and/or name of a file that contains sequence sample date information.\vspace{10pt}\\
\textbf{DEPENDENCIES} \vspace{5pt}\\
Model A00 must be used with tip-dating (\texttt{speciesdelimitation = 0}, \texttt{speciestree = 0}).
Checkpointing cannot be used. 
The tip-dating \texttt{locusrate} option must be used (\texttt{locusrate = 3 d d}). 
A global clock must be used. This is the default or can be set with \texttt{clock = 0}. \vspace{5pt}\\
\textbf{COMMENTS} \vspace{5pt}\\
See section \ref{datefile} for a description of the file format.
 \vspace{5pt}\\
\textbf{EXAMPLES} \vspace{5pt}\\
\textbf{\texttt{datefile = dates.txt}}\vspace{5pt}\\
\textbf{\texttt{datefile = /home/foo/seqDates.txt}}\vspace{10pt}\\
\noindent\textbf{{\large 2}} \\
\noindent\rule{\textwidth}{0.8pt} \\
\textbf{{\Large \texttt{locusrate = +d[(f f f s, s)]}}} \vspace{5pt}\\
\textbf{DESCRIPTION} \vspace{5pt}\\
Specifies the model of substitution rate variation among loci, as well
as the parameters, and the prior on the parameters of the model.
\vspace{5pt}\\
\textbf{VALUES} \vspace{5pt}\\
\textbf{\texttt{d+[(f f f s, s)]}}, specifies the model and
parameters. The variable d+ takes one of 4 possible values with
different numbers of additional variables depending on d+. The
permissible combinations are:
\begin{table}[H]
	\begin{tabular}{@{}lll@{}}
		\toprule
		\textbf{d+} & \textbf{{[}args{]}}  & \textbf{Description} \\ \midrule
		0 &  & Loci have same rate \\
		&  &  \\
		1 & f f f s & Locus rates variable and estimated \\
		& $\alpha_{\bar{\mu}} \,\, \beta_{\bar{\mu}} \,\, \alpha_{\mu_i}$ prior & \\
		2 & s & Locus rates variable and specified \\ 
		& Filename with locus rates & \\
		3 & $\alpha$ $\beta$ &  Loci have same rate which is estimated using tip dating\\ \bottomrule
	\end{tabular}
\end{table}
\noindent
where $\alpha_{\bar{\mu}}$ and $\beta_{\bar{\mu}}$ are the parameters
of the Gamma distribution prior on the average rate among loci
$\bar{\mu}$ and $\alpha_{\mu_i}$ is the parameter of the prior on the
rate at the $i$th locus given $\bar{\mu}$. The prior variable must be
either \texttt{iid} for a conditional iid (hierarchical) prior on
$\mu_i$ or \texttt{dir} for a Gamma-Dirichlet prior. The prior
variable is optional, if it is not specified \texttt{iid} is used.
\vspace{5pt}\\
\textbf{DEFAULT} \vspace{5pt}\\
\textbf{\texttt{0}} \vspace{5pt}\\
\textbf{DEPENDENCIES} \vspace{5pt}\\
If both the \texttt{prior} option for the variable \texttt{locusrate}
and the \texttt{prior} option
for the variable \texttt{clock} are set, they should take the same value.  
If the \texttt{locusrate = 3 d d}, the \texttt{datefile} option must be specified. \vspace{5pt} \\
\textbf{COMMENTS} \vspace{5pt}\\
The setting \texttt{locusrate = 0} (default) means that all loci have
the same mutation rate.  Setting \texttt{locusrate = 1 f f f s}
specifies a model with rate variation among loci in which rates are
estimated from the sequence data.  Including the first integer (1)
which species the model there are 5 arguments, the last one (prior)
being optional.  Parameters $\alpha_{\bar{\mu}}$ and
$\beta_{\bar{\mu}}$ specify the shape and rate parameters of the gamma
distribution for the mean rate across loci ($\bar\mu$), while
$\alpha_{\mu_i}$ and prior are used to specify the locus rates
($\mu_i$) given the mean rate ($\bar\mu$). The option prior can take
two values dir (for Gamma-Dirichlet rates for loci), and iid (for
conditional i.i.d.\ or hierarchical prior).  \texttt{locusrate = 2
	LocusRateFileName} specifies the fixed-rates model of locus-rate
variation \citep{Burgess2008}.  This is the strategy used by
\citet{Yang2002}, with the relative rates estimated by the distance to
an outgroup species.  The relative locus rates are listed in the file:
there should be as many numbers in the file, separately by spaces or
line returns, as the number of loci (nloci).  The program re-scales
those rates so that the average across all loci is 1 and then use
those relative rates as fixed constants.  Specifically the mean rate
across loci ($\bar{\mu}$) is assigned a gamma prior:
\begin{displaymath}
	\bar{\mu} = \mathrm{G}(\alpha_{\bar{\mu}},\beta_{\bar{\mu}}),
\end{displaymath}
with mean and variance
\begin{eqnarray}
	\textrm{Mean}(\bar{\mu}) & = & \frac{\alpha_{\bar{\mu}}}{\beta_{\bar{\mu}}}, \nonumber \\
	\textrm{Var}(\bar{\mu}) & = & \frac{\alpha_{\bar{\mu}}}{\beta_{\bar{\mu}}^2}. \nonumber 
\end{eqnarray}
When there are no fossil calibrations in the species tree, the rates
should all be relative.  In this case we suggest fixing
$\alpha_{\bar{\mu}}=\beta_{\bar{\mu}}=0$ (with 0 causing the program
to fix both parameters at $\infty$), and the program will fix the mean
rate across loci at $\bar\mu = 1$.  Otherwise one can use equal and
large values such as $\alpha_{\bar{\mu}}=100$ and
$\beta_{\bar{\mu}}=100$, so that $\bar\mu$ is nearly fixed at 1.
Given the mean rate $\bar\mu$, two priors are available to specify the
locus rates $\mu_i$, with $i = 1, 2, \cdots, L$, where $L$ is the
number of loci (\texttt{nloci}).  If \texttt{prior = dir} (for
Gamma-Dirichlet distribution of locus rates), the total rate
$L\bar\mu$, given the mean rate ($\bar\mu$), is partitioned into locus
rates ($\mu_i$), using the concentration parameter $\alpha_{\mu_i}$
The model and notation follow %\citet[][eq.~4]{Burgess2008} and
\citet[][eqs.~3-5]{dosReis2014}.  If \texttt{prior = iid} (for
conditional-i.i.d.\ or hierarchical prior of locus rates), the locus
rates ($\mu_i$) are i.i.d.\ given the mean rate ($\bar\mu$):
\begin{displaymath}
	\mu_i \sim \mathrm{G}(\alpha_{\mu_i},\alpha_{\mu_i}/\bar{\mu}).
\end{displaymath}
This model is described in \citet[][eq.~8]{Zhu2015} 
and is also
implemented in \textsc{mcmctree}.  In both the Gamma-Dirichlet and the
conditional i.i.d.\ models, parameter $\alpha_{\mu_i}$ is inversely
related to the extent of rate variation among loci, with a large
$\alpha_{\mu_i}$ meaning similar rates among loci.  If all loci are
noncoding, the rates are probably similar, so $\alpha_{\mu_i} =$ 10 or
20 may be reasonable, while for coding loci or exons,
$\alpha_{\mu_i} =$ 2 or 1 may be appropriate.  The $\alpha_{\mu_i}$
parameter may affect the estimates of the population size parameter
($\theta$) for the root node on the species tree \citep{Burgess2008}.
The $L$ locus rates ($\mu_i$) are parameters in the model.  If
$\alpha_{\bar{\mu}} > 0$, the mean rate ($\bar\mu$) is a parameter as
well.

The setting \texttt{locusrate = 3 d d}  means that all loci have
the same mutation rate, and the rate is estimated using tip dating.
The rate prior is $\Gamma(\alpha, \beta)$. 
The units of the prior should match the units of in the date file. 
For example, if the dates are specified in years, the rate should be in expected number of substitutions per year.
\vspace{5pt}\\
\textbf{EXAMPLES} \vspace{5pt}\\
\textbf{\texttt{locusrate = 0}} \vspace{5pt}\\
\textbf{\texttt{locusrate = 2 rates.txt}}\vspace{5pt}\\
\textbf{\texttt{locusrate = 1 0 0 2}} \vspace{5pt}\\
\textbf{\texttt{locusrate = 1 2 3 2 dir}} \vspace{5pt}\\
\textbf{\texttt{locusrate = 3 20 1000000}}\vspace{5pt}\\
\textbf{\texttt{locusrate = 1 1 1 1 iid}} \vspace{10pt}\\


\section{Input file format}
\subsection{Date File} \label{datefile} 
Each individual has an associated sample age. 
The \textbf{date file} assigns a date to the individual using the individual ID tag (described in the main bpp manual). 
The dates are in units before time present, so larger numbers are older.
Any desired time units (e.g. years, thousands of year, etc) can be used so long as the mutation rate prior is on the same time scale (e.g. expected substitutions per year, expected substitutions per thousand years).
For example, the following date file assigns \texttt{specimen1} an age 500, \texttt{specimen2} an age of 10000. The population name can be included before the caret \^{} symbol or it can be excluded. 
\begin{verbatim}
	A^specimen1 500
	specimen2 10000
	B^specimen3 45000
	B^specimen4 35000
\end{verbatim}

Each line should have one individual followed by the sample age.
Each individual must be included in the date file.
See the dates file \texttt{mammoth/dates.txt} in the \texttt{examples} subdirectory.


\section{Simulator}
Most of the options in the simulator remain the same. The only differences in the simulator options are related to specifying tip ages. 
We refer the reader to the main \texttt{bpp} manual for a more complete description of the control file options for simulation. 

\subsection{Control file} 
\texttt{datefile} specifies the name/path to the file containing the sequence ages. 
The ages are in units of expected number of substitutions, which is not the same as the inference program (see \ref{controlfilevariables} for more detail). 
The dates are assigned to populations.
The number of sample dates must match the number of individuals for each population.
For example, the below date file has 3 individuals from population A with sample ages of 0.001, 0.0016, and 0.001 in units of expected number of substitutions.
There are also two individuals from population B and two individuals from population C.
\begin{verbatim}
A 0.001
A 0.0016
A 0.001
B .0005
B .0002
C .0014
C .0012
\end{verbatim}
\texttt{seqDates} specifies the name/path to the output file for the date file with individual names with their sample dates.
The \texttt{bpp} simulator automatically names the indidivduals and sequences based on the population names. 
Since the individuals are not named in the input files for the simulator, the simulator must generate a file to match the dates to the individuals. Below is the \texttt{seqDates} file that corresponds to the \texttt{datefile} shown above.
\begin{verbatim}
	A^a1 0.001000
	A^a2 0.001000
	A^a3 0.001600
	B^b1 0.000200
	B^b2 0.000500
	C^c1 0.001200
	C^c2 0.001400
\end{verbatim}

\subsection{Using the simulator}
The \texttt{seqDates} and \texttt{datefile} options must be used together in the simulator.
A strict clock model (the default, which is equivalent to \texttt{clock = 1}) is required for simulating with tip dates. 
The simulator uses dates in units of expected substitutions, but the inference program uses dates in calendar time such as years or thousand years. 
The dates in the \texttt{seqDates} file can be converted to a unit of years by dividing by the per year substition rate.
This is required to use the simulator to generate data to use in the inference program.
For example, if we assume a substitution rate of $10^{-8}$ per year and use the same example as above, the \texttt{datefile} to use for inference would be
\begin{verbatim}
	A^a1 100000
	A^a2 100000
	A^a3 160000
	B^b1  20000
	B^b2  50000
	C^c1 120000
	C^c2 140000
\end{verbatim}
\texttt{bpp} does not generate a file with the dates in years or similar units. 
The user must prepare this file after running the simulator.
\bibliographystyle{natbib} \bibliography{/home/anna/bpp-manual/bpp-4-manual.bib}
\end{document}

